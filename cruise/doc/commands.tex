%
% $Id: commands.tex,v 1.4 2002/02/12 22:15:32 wolli Exp $
%
\chapter{Cruise-Kommandos}
\section{Allgemein}

\subsection{Source Referenz}

Jede Referenzangabe wird durch eine Referenz-Typ Angabe eingeleitet. 
Diese Typ Angabe setzt sich aus einem oder mehreren Buchstaben gefolgt
von einem Doppelpunkt zusammen. 

\begin{description}
\item[Dateiname:]
\begin{tabbing}
\hspace{10em}\=\\
Syntax:		\>f:\am{filename}\\
default-Wert:	\>aktuelle Datei\\
Beispiel:	\>\verb#f:~/.bashrc#\\
		\>\verb#f:/usr/local/share/readme.txt#\\
\end{tabbing}

\item[Zeile:]
\begin{tabbing}
\hspace{10em}\=\\
Syntax:		\>l:\am{absolute\_line}\\
		\>l:+\am{relative\_line}\\
		\>l:-\am{relative\_line}\\
default-Wert:	\>l:+1	(n�chste Zeile)\\
Beispiel:	\>\verb#l:4  #(Zeile 4)\\
		\>\verb#l:+2 #(die �bern�chste Zeile)\\
		\>\verb#l:-1 #(die vorige Zeile)\\
\end{tabbing}


\item[Spalte:]
\begin{tabbing}
\hspace{10em}\=\\
Syntax:		\>c:\am{absolute\_column}\\
		\>c:\#\am{field}\\
default-Wert:	\>c:\#1  (erstes Feld)\\
Beispiel:	\>\verb#c:3  #(dritte Spalte)\\
		\>\verb+c:#4  +(viertes Feld)\\
\end{tabbing}
	
\item[Spaltentrenner:]
\begin{tabbing}
\hspace{10em}\=\\
Syntax:		\>cd:\am{column\_delimiter}\\
default-Wert:	\>cd:ws  (whitespace: Tabulator und Leerzeichen)\\
Beispiel:	\>\verb#cd:","  #(Komma)\\
		\>\verb#cd:" "  #(ein Leerzeichen)\\
\end{tabbing}
Idee: Regul�rer Ausdruck als Spaltentrennerdefinition !


\item[Marke:]
\begin{tabbing}
\hspace{10em}\=\\
Syntax:		\>m:\am{mark}\\
default-Wert:	\>-  (kein default-Wert)\\
Beispiel:	\>\verb#m:mark\_1#\\
\end{tabbing}


\item[URL:]
\begin{tabbing}
\hspace{10em}\=\\
Syntax:		\>u:\am{url}\\
default-Wert:	\>-  (kein default-Wert)\\
Beispiel:	\>\verb#u:ftp://ftp.suse.de/misc/readme.txt#\\
\end{tabbing}

\item[Zeiger oben:]
\begin{tabbing}
\hspace{10em}\=\\
Syntax:		\>\^{ }\\
default-Wert:	\>-  (kein default-Wert)\\
Beispiel:	\>\verb#^#\\
\end{tabbing}


\item[Zeiger unten:]
\begin{tabbing}
\hspace{10em}\=\\
Syntax:		\>v\\
default-Wert:	\>-  (kein default-Wert)\\
Beispiel:	\>\verb#v#\\
\end{tabbing}

\end{description}

	
\subsubsection{Anwendungs-Beispiel 1}

Annahme: das cruise Token w�re "\$c\$".
Mit folgendem Beispiel wird auf die Zahl hinter 'MAX\_NUM' referenziert.

\begin{verbatim}
$c$ listbox {214 220 417 420} c:#3
#define MAX_NUM 214
\end{verbatim}
\begin{itemize}
\item c:\#3 verweist auf das dritte Feld
\item Dateiname wurde nicht angegeben, also aktuelle Datei
\item Zeile wurde nicht angegeben, also n�chste Zeile
\item Spaltentrenner wurde nicht angegeben, also whitespace
\end{itemize}

\subsubsection{Anwendungs-Beispiel 2}

Annahme: das cruise Token w�re "\$c\$".
Mit folgendem Beispiel wird auf die Zahl hinter dem 
Gleicheitszeichen referenziert.


\begin{verbatim}
$c$ listbox {3.14 3.141 3.1415 3.14159} c:#4 l:+2
// declaration and definition of PI
double pi = 3.1415 ;
\end{verbatim}
\begin{itemize}
\item c:\#4 verweist auf das vierte Feld
\item l:+2 verweist auf die �bern�chste Zeile
\end{itemize}













