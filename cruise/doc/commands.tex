%
% $Id: commands.tex,v 1.5 2002/03/24 01:03:38 wolli Exp $
%
\chapter{Cruise-Kommandos}
\section{Optionen}

\subsection{Source Referenz}

Jede Referenzangabe wird durch eine Referenz-Typ-Angabe eingeleitet. 
Die Typ-Angabe wird in Form einer Option dem Cruise-Kommando mitgeteilt. 


\begin{description}
\item[Dateiname:]
\begin{tabbing}
\hspace{10em}\=\\
Syntax:		\>-f\am{filename}\\
default-Wert:	\>aktuelle Datei\\
Beispiel:	\>\verb#-f ~/.bashrc#\\
		\>\verb#-f /usr/local/share/readme.txt#\\
\end{tabbing}
Mit dieser Option wird wird auf eine bestimmte Datei verwiesen. Existiert diese Datei nicht, so wird eine Fehlermeldung
ausgegeben.

\item[Zeile:]
\begin{tabbing}
\hspace{10em}\=\\
Syntax:		\>-l \am{absolute\_line}\\
		\>-l +\am{relative\_line}\\
		\>-l -\am{relative\_line}\\
default-Wert:	\>-l +1	(n�chste Zeile)\\
Beispiel:	\>\verb#-l 4  #(Zeile 4)\\
		\>\verb#-l +2 #(die �bern�chste Zeile)\\
		\>\verb#-l -1 #(die vorige Zeile)\\
\end{tabbing}
Mit dieser Option wird auf eine Zeile verwiesen. Ist die referenzierte Zeilennummer gr��er, als die Anzahl der Zeilen in
der referenzierten Datei so wird auf die letzte Zeile verwiesen. Ist die referenzierte Zeilennummer negativ, so wird auf
die erste Zeile der Datei verwiesen.


\item[Spalte:]
\begin{tabbing}
\hspace{10em}\=\\
Syntax:		\>-c  \am{absolute\_column}\\
		\>-c \#\am{field}\\
default-Wert:	\>-c \#1  (erstes Feld)\\
Beispiel:	\>\verb#-c 3  #(dritte Spalte)\\
		\>\verb+-c #4 +(viertes Feld)\\
\end{tabbing}
Mit dieser Option wird auf eine Spalte verwiesen. Ist die referenzierte Spaltennummer gr��er, als die Anzahl der Spalten in
der referenzierten Zeile so wird auf die letzte Spalte verwiesen. Ist die referenzierte Spaltennummer negativ, so wird auf
die erste Spalte der Zeile verwiesen.

	
\item[Spaltentrenner:]
\begin{tabbing}
\hspace{10em}\=\\
Syntax:		\>-cd \am{column\_delimiter}\\
default-Wert:	\>-cd ws  (whitespace: Tabulator und Leerzeichen)\\
Beispiel:	\>\verb#-cd ","  #(Komma)\\
		\>\verb#-cd " "  #(ein Leerzeichen)\\
\end{tabbing}


Idee: Regul�rer Ausdruck als Spaltentrennerdefinition !


\item[Marke:]
\begin{tabbing}
\hspace{10em}\=\\
Syntax:		\>-m \am{mark}\\
default-Wert:	\>-  (kein default-Wert)\\
Beispiel:	\>\verb#-m mark_1#\\
\end{tabbing}


\item[URL:]
\begin{tabbing}
\hspace{10em}\=\\
Syntax:		\>-u \am{url}\\
default-Wert:	\>-  (kein default-Wert)\\
Beispiel:	\>\verb#-u ftp://ftp.suse.de/misc/readme.txt#\\
\end{tabbing}

\item[Zeiger oben:]
\begin{tabbing}
\hspace{10em}\=\\
Syntax:		\>\^{ }\\
default-Wert:	\>-  (kein default-Wert)\\
Beispiel:	\>\verb#^#\\
\end{tabbing}


\item[Zeiger unten:]
\begin{tabbing}
\hspace{10em}\=\\
Syntax:		\>v\\
default-Wert:	\>-  (kein default-Wert)\\
Beispiel:	\>\verb#v#\\
\end{tabbing}

\end{description}

	
\subsubsection{Anwendungs-Beispiel 1}

Annahme: das cruise Token w�re "\$c\$".
Mit folgendem Beispiel wird auf die Zahl hinter 'MAX\_NUM' referenziert.

\begin{verbatim}
$c$ menubutton -c #3 -ol {110 220 214 889}
#define MAX_NUM 214
\end{verbatim}
\begin{itemize}
\item -c \#3 verweist auf das dritte Feld
\item Dateiname wurde nicht angegeben, also aktuelle Datei
\item Zeile wurde nicht angegeben, also n�chste Zeile
\item Spaltentrenner wurde nicht angegeben, also whitespace
\end{itemize}

\subsubsection{Anwendungs-Beispiel 2}

Annahme: das cruise Token w�re "\$c\$".
Mit folgendem Beispiel wird auf die Zahl hinter dem 
Gleicheitszeichen referenziert.


\begin{verbatim}
$c$ menubutton -c #4 -l +2 -ol {3.14 3.141 3.1415 3.14159} 
// declaration and definition of PI
double pi = 3.1415 ;
\end{verbatim}
\begin{itemize}
\item -c \#4 verweist auf das vierte Feld
\item -l +2 verweist auf die �bern�chste Zeile
\end{itemize}


\subsection{Textoptionen}

Mit einer Textoption werden Textinformationen f�r den Benutzer an das jeweilige Kommando �bergeben.

\begin{description}
\item[Beschriftung:]
\begin{tabbing}
\hspace{10em}\=\\
Syntax:		\>-t\am{text}\\
default-Wert:	\>-\\
Beispiel:	\>\verb#-t "Neuer Wert:"#\\

\end{tabbing}
Der mit der Option angegebene Text wird zur Beschriftung des Bedienelementes verwendet.

\item[Hilfetext:]
\begin{tabbing}
\hspace{10em}\=\\
Syntax:		\>-h\am{text}\\
default-Wert:	\>-\\
Beispiel:	\>\verb#-h "W�hlen sie einen Wert zwischen 0 und 255 !"#\\
		\>\verb#-h $HELPTEXT#\\
\end{tabbing}
Wird mit einem Kommando ein Bedienelement mit Hilfeknopf erzeugt, so wird mit diesem Kommando der Text �bergeben, der im Textfenster des Hilfeknopfs erscheinen soll.


\item[Parameterliste:]
\begin{tabbing}
\hspace{10em}\=\\
Syntax:		\>-ol\am{text1 text2 text3 text4}\\
default-Wert:	\>-\\
Beispiel:	\>\verb#-ol {9600 19200 57200}#\\

\end{tabbing}
Kommandos welche eine Liste von ausw�hlbaren Texten anbieten erhalten mit dieser Option die Auswahlliste.

\end{description}









