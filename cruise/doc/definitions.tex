%
% $Id: definitions.tex,v 1.1 2001/08/03 20:17:16 klauko70 Exp $
%
\part{Definitionen}
\subsection{Ersteller}
Der Ersteller ist diejenige Person, die in ein Sourcefile \cruise-Zeilen einf�gt.

\subsection{Benutzer}
Der Benutzer ist diejenige Person, die mittels Mauszeiger und/oder Tastatur 
mit dem vom \cruise-Interpreter erzeugten Benutzerinterface interagiert.

\subsection{Sourcefile}
Das Sourcefile ist die Datei, in die der Ersteller \cruise-Zeilen einf�gt, um eine
\cruise-Funktionalit�t zu erreichen. Ein Sourcefile kann beispielsweise ein
bash-Script, ein Assmbler-Code, ein C-Source-Code oder ein Tex-File sein.

\subsection{\cruise-Zeile}
Eine \cruise-Zeile wird in das Sourcefile eingef�gt. Damit der Zielprozess die
\cruise-Zeile unbeachtet l��t, mu� die \cruise-Zeile als Kommentar in das Sourcefile eingef�gt werden. Die Einleitung dieses Kommentars richtet sich nach Art des
Zielprozess'

\subsection{Zielprozess}
Der Zielprozess ist das Programm f�r den das Sourcefile eigentlich gedacht ist.


