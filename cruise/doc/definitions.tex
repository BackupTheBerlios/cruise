%
% $Id: definitions.tex,v 1.2 2001/08/11 17:36:24 klauko70 Exp $
%
\chapter{Definitionen}

\section{\cruise}
Comfortable Remark Based User Interface for Source Editing

\section{Interpretation}
Die Interpretation ist die Abarbeitung der einzelnen \cruise-Zeilen. Diese findet je Sourcefile genau einmal statt und zwar unmittelbar nach dem Start von \cruise.

\section{Interaktion}
Die Interaktion ist die Einwirkung mit Hilfe von Mauszeiger und/oder Tastatur, auf das mit dem \cruise-Interpreter erzeugte Benutzerinterface.

\section{Ersteller}
Der Ersteller ist diejenige Person, die in ein Sourcefile \cruise-Zeilen einf�gt.

\section{Benutzer}
Der Benutzer ist diejenige Person, die mit dem erzeugten Benutzerinterface interagiert.

\section{Sourcefile}
Das Sourcefile ist die Datei, in die der Ersteller \cruise-Zeilen einf�gt, um eine
\cruise-Funktionalit�t zu erreichen. Ein Sourcefile kann beispielsweise ein
bash-Script, ein Assmbler-Code, ein C-Source-Code oder ein Tex-File sein.

\section{\cruise-Zeile}
Eine \cruise-Zeile wird in das Sourcefile eingef�gt. Damit der Zielprozess die
\cruise-Zeile unbeachtet l��t, mu� die \cruise-Zeile als Kommentar in das Sourcefile eingef�gt werden. Die Einleitung dieses Kommentars richtet sich nach Art des
Zielprozess'

\section{Zielprozess}
Der Zielprozess ist das Programm f�r den das Sourcefile eigentlich gedacht ist.

\section{Ausgabe}
Die Ausgabe ist die Datei, die nach Interpretation des Sourcefiles und anschliessender Interaktion durch den Benutzer von \cruise\ erzeugt wird.


%% klauko70 -> a nice chart would help to understand the above






